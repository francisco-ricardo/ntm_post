\documentclass[a4paper]{article}

\usepackage{amssymb}
\usepackage[shortlabels]{enumitem}
\usepackage{indentfirst}

\usepackage{listings}

\lstdefinestyle{mystyle}{
    basicstyle=\ttfamily\footnotesize,
    breakatwhitespace=false,
    breaklines=true,
    captionpos=b,
    keepspaces=true,
    showspaces=false,
    showstringspaces=false,
    showtabs=false,
    tabsize=2
}
\lstset{style=mystyle}


\title{
    Data Structure Review: Graphs\\
}

\author{
    \small{Author: Francisco Ricardo Taborda Aguiar}\\
}

\date{\today}

\begin{document}

    \maketitle

    A graph is a representation of relationship between objects.
    The set of objects is called vertex (or nodes). The connections 
    between the vertex are called edges (or arcs).
    In an abstract point of view, a graph \emph{G} is a set of vertex 
    \emph{V} and a collection \emph{E} of pairs of vertex (the edges).

    representação matemática
    história
    aplicações práticas (enfatizar neo4j)
    código fonte

        \cite{goodrich:2014}


    \section*{Conclusion}

    \bibliographystyle{ieeetr}
    \bibliography{graphs} 

\end{document}
