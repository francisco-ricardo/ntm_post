\documentclass[a4paper]{article}

\usepackage{amssymb}
\usepackage[shortlabels]{enumitem}
\usepackage{indentfirst}

\usepackage{listings}

\lstdefinestyle{mystyle}{
    basicstyle=\ttfamily\footnotesize,
    breakatwhitespace=false,
    breaklines=true,
    captionpos=b,
    keepspaces=true,
    showspaces=false,
    showstringspaces=false,
    showtabs=false,
    tabsize=2
}
\lstset{style=mystyle}


\title{
    Data Structure Review: Graphs\\
}

\author{
    \small{Author: Francisco Ricardo Taborda Aguiar}\\
}

\date{\today}

\begin{document}

    \maketitle

    A graph is a way to represent relationships between pairs 
    of objects \cite{goodrich:2014} or entities.
    A graph can be defined by a set of vertices and a collection 
    of edges.
    Vertices represent entities in graph, whereas edges represent 
    relationships between those entities \cite{xia:2021}.
    In an abstract point of view, a graph \emph{G} is a set of vertex 
    \emph{V} and a collection \emph{E} of pairs of vertex (the edges).

    The graph may be directed, when the edges are ordered pairs 
    \emph{(u, v)} of vertices, with \emph{u} preceding \emph{v}.
    The graph is undirected, when the edges are unordered pairs of 
    vertices, also represented as \emph{(u, v)}.

    The two vertices connected by an edge are called endpoints.
    Adjacent vertices are two vertices that are joined by an edge.
    Adjacent edges are two edges that have an endpoint in common.
    A vertex joined to itself by an edge is called loop.

    An edge is incident to a vertex if the vertex is an endpoint of
    the edge.

    The degree of a vertex \emph{v} (\emph{deg{v}}) corresponds to 
    the number of the incident edges to the vertex \emph{v}.
    The input degree (\emph{indeg(v)}) of a vertex \emph{v} consists 
    of the number of incident edges in \emph{v}.
    The output degree (\emph{outdeg(v)}) of a vertex \emph{v} is the 
    number of incident edges from \emph{v}.

    If \emph{G} is a graph with \emph{m} edges, then:    
    \[ \sum_{v \in G} deg(v) = 2m \]

    If \emph{G} is a directed graph with \emph{m} edges, then:
    \[ \sum_{v \in G} indeg(v) = \sum_{v \in G} outdeg(v) = m\]


 
    


    história
    aplicações práticas (enfatizar neo4j)
    código fonte

        


    \section*{Conclusion}

    \bibliographystyle{ieeetr}
    \bibliography{graphs} 

\end{document}



https://math.libretexts.org/Bookshelves/Combinatorics_and_Discrete_Mathematics/Combinatorics_and_Graph_Theory_(Guichard)/05%3A_Graph_Theory/5.01%3A_The_Basics_of_Graph_Theory