\documentclass[a4paper]{article}

\usepackage{amssymb}
\usepackage[shortlabels]{enumitem}
\usepackage{indentfirst}

\usepackage{listings}

\lstdefinestyle{mystyle}{
    basicstyle=\ttfamily\footnotesize,
    breakatwhitespace=false,
    breaklines=true,
    captionpos=b,
    keepspaces=true,
    showspaces=false,
    showstringspaces=false,
    showtabs=false,
    tabsize=2
}
\lstset{style=mystyle}


\title{
    Data Structure Review: Graphs\\
}

\author{
    \small{Author: Francisco Ricardo Taborda Aguiar}\\
}

\date{\today}

\begin{document}

    \maketitle

    A graph can be defined by two sets: vertex set and edge set. 
    Vertices represent entities in graph, whereas edges represent 
    relationships between those entities \cite{xia:2021}.
    In an abstract point of view, a graph \emph{G} is a set of vertex 
    \emph{V} and a collection \emph{E} of pairs of vertex (the edges).

    The graph may be directed, when the edges are ordered pairs 
    \emph{(v, w)} of vertices, where \emph{v} is the tail and \emph{w} 
    is the head of the edge.
    The graph is undirected, when the edges are unordered pairs of 
    vertices, also represented as \emph{(v, w)}.


    representação matemática
    história
    aplicações práticas (enfatizar neo4j)
    código fonte

        \cite{goodrich:2014}


    \section*{Conclusion}

    \bibliographystyle{ieeetr}
    \bibliography{graphs} 

\end{document}



https://math.libretexts.org/Bookshelves/Combinatorics_and_Discrete_Mathematics/Combinatorics_and_Graph_Theory_(Guichard)/05%3A_Graph_Theory/5.01%3A_The_Basics_of_Graph_Theory