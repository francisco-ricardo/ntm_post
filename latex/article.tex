\documentclass[a4paper]{article}

\usepackage[shortlabels]{enumitem}


\title{
    Aula 3 - Resenha dos artigos\\
}

\author{
    {EMAN7015 - Transforma\c c\~ao Digital de Sistemas de Manufatura}\\
    \small{Professor: Fernando Deschamps}\\
    \small{Aluno: Francisco Ricardo Taborda Aguiar}\\    
}

\date{\today} 

\begin{document}

    \maketitle

    Until a few years ago the development of Machine Translation (MT) 
    systems was mostly implemented by statistical techniques, know as 
    Statistical Machine Translation (SMT).
    Those techniques became able to extract implicit information from
    \emph{copora bil\'ingues} , \cite{brown:1993}.
    However, the authors of \cite{maruf:2021} noted that the SMT features
    were intrinsic to the tenchologies and this made the tenchologies
    inflexible.

    TODO

    At\'e h\'a alguns anos o desenvolvimento de sistemas de tradu\c c\~ao autom\'atica 
    (Machine Translation, ou MT) era, em sua maioria, implementado atrav\'es de t\'ecnicas 
    estat\'isticas, conhecidas como Tradutor de M\'aquinas Estat\'isticas 
    (Statistical Machine Translation, ou SMT).
    Estas t\'ecnicas tornaram capaz a extra\c c\~ao de informa\c c\~oes impl\'icitas a partir de
    \emph{copora bil\'ingues}, \cite{brown:1993}. Por\'em, em \textcite{maruf:2021}, verifica-se que 
    as caracter\'isticas do \emph{SMT} eram intr\'insecas \`a tecnologia e isto tornava estas 
    t\'ecnicas inflex\'iveis. V\'arios trabalhos t\^em impulsionado o emprego de redes 
    neurais no Processamento de Linguagem Natural (NLP), conforme pode ser verificado em 
    \textcite{goldberg:2016}. Por\'em, a maioria das abordagens consistiam em utilizar redes neurais 
    como sendo componentes em sistemas \emph{STM} tradicionais, apenas substuindo algumas 
    partes da arquitetura, como pode ser notado em \textcite{stahlberg:2020}.
    
    O processo de \emph{MT} tem avan\c cado para o uso de sistemas conhecidos como Tradu\c c\~ao 
    Autom\'atica Neural (Neural Machine Translation, ou NMT).
    Estes sistemas s\~ao baseados em redes neurais que fazem a tradu\c c\~ao das senten\c cas.
    \emph{NMT} vem se tornando o paradigma dominante para a pesquisa e o desenvolvimento 
    de sistemas de \emph{MT} e tem sido utilizado em sistemas de produ\c c\~ao, como na Google
    \cite{wu:2016}, por exemplo.
    
    O uso de Redes Neurais Profundas (Deep Neural Network, ou DNN) no processo de \emph{MT}
    tem se destacado atrav\'es dos modelos Sequ\^encia a Sequ\^encia, que fazem o uso de 
    Redes Neurais Recorrentes (Recurrent Neural Networks, ou RNNs).
    A arquitetura b\'asica destes modelos consiste em um codificador RNN que percorre cada 
    token da senten\c ca de origem e gera um vetor de estado com tamanho fixo.
    Na sequ\^encia, um decodificador RNN gera a senten\c ca de destino, um token de cada vez,
    a partir do vetor de estado, como pode ser visto em \textcite{sutskever:2014}.
    \textcite{kalchbrenner:2016} utilizaram uma Rede Neural Convolucional (Convolutional 
    Neural Network, ou CNN). O codificador e o decodificador s\~ao conectados e as sequ\^encias 
    temporais s\~ao preservadas. Os autores introduziram um mecanismo para tratar das 
    diferen\c cas entre o comprimento da origem e o comprimento do destino.
    
    Outra arquitetura utilizada em sistemas NMT \'e o modelo Transformador, introduzido por
    \textcite{vaswani:2017attention}. Este modelo dispensa o uso de recorr\^encia e convolu\c c\~ao
    e utiliza um mecanismo conhecido como Aten\c c\~ao.
    Este mecanismo baseia-se em uma fun\c c\~ao que mapeia perguntas e um conjunto de pares do 
    tipo \emph{chave-valor} para uma sa\'ida. As perguntas, as chaves, os valores e a sa\'ida
    correspondem a vetores. A sa\'ida \'e calculada atrav\'es de uma soma ponderada dos valores,
    aonde o peso associado para cada valor \'e calculado atrav\'es de uma fun\c c\~ao de 
    compatibilidade entre a pergunta e a chave correspondente.
    
    A Tradução Automática Neural poderia ser aplicada para a conversão dos dialetos RS274-D 
    para as Funções Canônicas de Usinagem. Porém, vale observar que isto demandaria a obtenção de 
    um acervo de programas escritos nos diversos dialetos para vialibilizar os processos de 
    treinamento e teste da rede neural.
    





\bibliographystyle{ieeetr}
\bibliography{article} 

\end{document}
